% ----------------------------------------------------------
% EPÍGRAFE

%Epígrafe: Elemento opcional e sem título em que o (a) autor (a) apresenta uma citação relacionada ao assunto tratado no trabalho. Deve ser elaborada conforme a ABNT-NBR 10520 (Citações). As citações de até três linhas devem estar entre aspas duplas e as citações com mais de três linhas devem ser destacadas com recuo de 4 cm da margem esquerda, com letra menor que a do texto e sem as aspas.
% ---------------------------------------------------------
\vspace*{10cm}
\begin{citacao}
Exemplo de epígrafe com mais de 3 linhas: Elemento opcional e sem título em que o (a) autor (a) apresenta uma citação relacionada ao assunto tratado no trabalho. Deve ser elaborada conforme a ABNT-NBR 10520 (Citações). As citações de até três linhas devem estar entre aspas duplas e as citações com mais de três linhas devem ser destacadas com recuo de 4 cm da margem esquerda, com letra menor que a do texto e sem as aspas. (ABNT, 2002).
\end{citacao}

    \vspace*{5cm}
	
		''Exemplo de epígrafe com até 3 linhas, geralmente citando pensador/filosofo: Texto texto texto texto texto texto texto texto texto texto texto texto texto texto texto texto texto texto texto texto texto texto.'' (Fulanum; Cicranum, 1625).
	
\newpage